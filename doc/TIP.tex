
\documentclass{article}

\usepackage{float}
\usepackage{textcomp}
\usepackage{polski}
\usepackage[utf8]{inputenc}
\usepackage[T1]{fontenc}
\usepackage{graphicx}
\usepackage{listings}
\graphicspath{ {res/} }



\usepackage[top=2cm, bottom=2cm, left=3cm, right=3cm]{geometry}

\makeatletter

\newcommand{\linia}{\rule{\linewidth}{0.5mm}}
\renewcommand\thesection{\arabic{section}}

\renewcommand{\maketitle}{\begin{titlepage}
		
		\vspace*{1cm}
		
		\begin{center}\small
			
			Politechnika Poznańska\\
			
			Wydział Elektryczny\\
			
			Kierunek Informatyka
			
		\end{center}
		
		\vspace{2cm}
		
		\linia
		
		\begin{center}
			
			\LARGE {\textbf{KakaduVoIP} \\[1cm]}
			\begin{figure}[h]
				\centering
				\includegraphics[scale=0.2]{Kakadu}
			\end{figure}
		  \LARGE {Projekt zespołowy  \\ Telefonia IP}
			
		\end{center}
		
		\linia
		
		\vspace{3cm}
		
		\begin{flushright}
			
			\begin{minipage}{15cm}
				
				\textit{ \Large{Autorzy:}}\\
				
				\normalsize{\@author} \par
				
			\end{minipage}
			
		\end{flushright}
		
		\vspace*{\stretch{6}}
		
		\begin{center}
			
			\@date
			
		\end{center}
		
	\end{titlepage}%
	
}

\makeatother

\author{\Large{Szymon Zieliński \\[0.2cm] szymon.r.zielinski@gmail.com \\[0.2cm] 126812\\[0.4cm] Oskar Rutkowski \\[0.2cm] oskar.rutkowski@student.put.poznan.pl \\[0.2cm] 126845} }


\begin{document}
	
	\pagenumbering{gobble}
	\maketitle
	\newpage
	\tableofcontents
	\newpage
	\pagenumbering{arabic}
	
	\section{Charakterystyka ogólna projektu}
	\paragraph*{} Aplikacja pozwala na komunikowanie się użytkowników za pomocą mikrofonu oraz głośników w czasie rzeczywistym. Aplikacja jest stworzona w języku Java oraz działa w oparciu o architekturę klient-serwer. Użytkownik po uruchomieniu, jest proszony o podanie nazwy użytkownika oraz adresu serwera  do którego chce się połączyć. Będąc połączonym, użytkownik ma możliwość stworzenia własnego pokoju konferencyjnego lub dołączenia do istniejącego już pokoju. Aby połączyć się z pokojem wybranym z listy, użytkownik musi podać hasło pokoju. Po podaniu prawidłowego hasła, użytkownik ma możliwość rozmowy przy pomocy naciśniętego przycisku lub zmianę metody komunikacji na opcję mówienia ciągłego. Użytkownik ma możliwość wyciszenia swojego mikrofonu lub głośników, jak też innego użytkownika, znajdującego się w pokoju.
	\section{Architektura systemu}
	\paragraph*{} System KakaduVoIP jest oparty o model architektury klient-serwer. Model ten umożliwia podział zadań, serwer zapewnia usługi dla klientów, a klienci zgłaszają żądania obsługi. Serwer składuje informacje o pokojach w plikach lokalnych.
	\section{Wymagania}
	\paragraph*{} W tym paragrafie zostaną omówione wymagania aplikacji funkcjonalne oraz niefunkcjonalne.
	\subsection{Wymagania funkcjonalne}
	\begin{itemize}
		\item użytkownik łączy się z systemem jedynie za pomocą nazwy użytkownika oraz adresu serwera,
		\item użytkownik może zmieniać ustawienia aplikacji względem możliwości komunikowania się, tj. mówienie przez naciśnięcie przycisku lub mówienie ciągłe,
		\item użytkownik może utworzyć pokój konferencyjny, nadając mu nazwę i hasło dostępu do pokoju oraz tworząc hasło administracyjne, stając się administratorem tego pokoju,
		\item administrator pokoju, który stworzył pokój, może usunąć pokój konferencyjny,
		\item administrator, który założył pokój, może wyrzucić innych użytkowników z pokoju.
		\item użytkownik może dołączyć do pokoju wybranego z listy aktualnie dostępnych pokoi na serwerze, podając hasło dostępu do pokoju,
		\item użytkownik w pokoju konferencyjnym może opuścić pokój,
		\item użytkownik w pokoju konferencyjnym może wyciszyć innego użytkownika, aby go nie słyszeć,
		\item użytkownik w pokoju konferencyjnym może komunikować się z innymi użytkownikami za pomocą przycisku lub mówienia ciągłego,
		\item użytkownik w pokoju konferencyjnym może wyciszyć swój mikrofon lub wyłączyć używanie głośników,
		
	\end{itemize}
	\subsection{Wymagania niefunkcjonalne}
	\begin{itemize}
		\item nowo utworzony pokój nie potrzebuje administratora pokoju będącego aktualnie w pokoju,
		\item pokój może być usunięty przez administratora pokoju dzięki podaniu hasła dostępu do pokoju oraz hasła administracyjnego,
		\item po usunięciu pokoju przez administratora pokoju, wszyscy użytkownicy zostają automatycznie przekierowani do głównej strony aplikacji,
		\item jeden użytkownik może być administratorem tylko jednego pokoju,
		\item długo nieużywane pokoje zostają automatycznie usunięte,
		\item aplikacja klienta oraz serwer napisane w języku Java,
		\item aplikacja wieloplatformowa,
		\item szyfrowanie dźwięku,
		\item szyfrowanie zapytań do serwera,
		\item do połączenia z serwerem nie jest potrzebna rejestracja użytkownika, jedynym wymogiem jest podanie nazwy użytkownika oraz adres serwera,
		\item do składowania haseł zabezpieczających pokoje, posłużą pliki lokalne składowane na serwerze,
		\item hasła dostępu do pokoju oraz hasła administracyjne będą poddane funkcji haszującej,
		\item interfejs stworzony w języku polskim,
		\item do sprawnego działania aplikacji, niezbędne są słuchawki lub głośniki oraz mikrofon.
	\end{itemize}
	\section{Narzędzia, środowiska, biblioteki, kodeki}
	\paragraph*{} W niniejszym paragrafie zostaną opisane wykorzystane w aplikacji narzędzia, środowiska, biblioteki oraz kodeki.
	\paragraph*{} Głównym językiem wykorzystanym do napisania aplikacji jest język Java w wersji 8 Update 162. Wykorzystana jest technika Maven Project, dzięki której możliwa jest praca w różnych środowiskach programistycznych takich jak Intellij oraz Eclipse Oxygen.
	\paragraph*{} Wykorzystane frameworki oraz biblioteki:
	\begin{itemize}
		\item Apache MINA - Open Source Framework, służący do budowania aplikacji sieciowych,
		\item Java Media Framework - technologia umożliwiająca wstawianie multimediów do aplikacji napisanych w języku Java oraz transmisję audio za pomocą protokołu RTP,
		\item Protocol Buffers - narzędzie do tworzenia własnych protokołów,
		\item java.security.MessageDigest - biblioteka wykorzystana do nakładania funkcji skrótu na hasła,
		\item javax.security oraz javax.crypto - biblioteki wykorzystane do szyfrowania.
	\end{itemize}
	\paragraph*{} Hasła wraz z potrzebnymi informacjami o pokojach oraz użytkownikach są przetrzymywane w plikach lokalnych przechowywanych po stronie serwera.
	\paragraph*{} Użyte kodeki, zawarte w JMF:\\ MPEG Layer II Audio (.mp2) [MPEG layer 2 audio ]
	\section{Protokoły}
	\paragraph*{} Jednymi z najważniejszych protokołów użytych są:
	\begin{itemize}
		\item Warstwa aplikacji:
		\begin{itemize}
			\item RTP - protokół transmisji w czasie rzeczywistym - do przesyłania dźwięku,
			\item RTCP (ang. Real Time Control Protocol) - protokół sterujący używany do okresowej transmisji pakietów kontrolnych do wszystkich uczestników sesji. Pozwala monitorować dostarczanie danych przez protokół RTP i transport zwrotnej informacji odnośnie jakości transmisji,
			\item protokoły własne - protokoły do komunikacji z serwerem:
		
		
			\item protokół informacji o pokoju:\\ \\
				message Room \{\\
				required string room\_name = 1;\\
				required string owner = 2;\\
				\}\\
				\\
				Protokół informacji o pokoju zawiera niezbędne pola do identyfikacji kontkretnych pokoi, takie jak nazwa pokoju oraz właściciel pokoju (użytkownik, który stworzył konkretny pokój).
			\item protokół żądania klienta podłączenia do serwera: \\ \\
				message LoginToServerRequest \{\\
				required string nick = 1;\\
				\}\\
				\\	
				Protokół żądania klienta podłączenia do serwera zawiera pole nick, wskazujące na nazwę użytkownika, chcącego podłączyć się do serwera.
			\item protokół odpowiedzi na żądanie klienta podłączenia do serwera: \\ \\
				message LoginToServerResponse \{\\
				required StatusCode status = 1;\\
				repeated Room roomList = 2;\\
				\}\\
				\\
				Protokół odpowiedzi, zwraca klientowi status żądania oraz listę dostępnych pokoi, przy poprawnym połączeniu się z serwerem.
			\item protokół opuszczenia serwera: \\
				message LeaveServerRequest\{\\
				required string nick = 1;\\
				\}\\
				\\
				Protokół opuszczenia serwera zawiera pole nick, wskazujący na użytkownika, który chce opuścić serwer.
			\item protokół odpowiedzi na opuszczenie serwera:\\
				message LeaveServerResponse \{\\
				required StatusCode status = 1;\\
				\}\\
				\\
				Protokół odpowiedzi zawiera pole status, informujące o pomyślnej lub negatywnej odpowiedzi na żądanie.
			\item protokół żądania zadrządzania pokojem:\\
				message ManageRoomRequest \{\\
					required ManageRoomEnum manageRoomEnum = 1;\\
					required string nick = 2;\\
					required string roomName = 3;\\
					optional string password = 4;\\
					optional string adminPassword = 5;\\
					optional string otherUserNick = 6;\\
				\}\\
				\\
				Protokół zarządzania pokojem służy do zarządzania pokojem. Niezbędne pola manageRoomEnum, nick oraz roomName wskazują kolejno na: akcje związane z zarządzaniem pokojem, nazwę użytkownika, który chce zarządzać oraz nazwę pokoju, który ma być zarządzany. Dodatkowe pola wskazują na hasło dołączenia do pokoju, hasło administracyjne, służące do usuwania pokoju przez administratora oraz nazwa innego użytkownika, w przypadku wyciszania lub usuwania z pokoju.
			\item protokół odpowiedzi na żądanie zarządzania pokojem:
				message ManageRoomResponse \{\\
					required StatusCode status = 1;\\
					optional string keyToRoom = 2;\\
				\}\\
				\\
				Protokół odpowiedzi zawiera status odpowiedzi oraz opcjonalny klucz szyfrowania.
			\item typ wyliczeniowy do zarządzania pokojem:\\
				enum ManageRoomEnum \{\\
					CREATE\_ROOM = 1;\\
					DELETE\_ROOM = 2;\\
					JOIN\_ROOM = 3;\\
					LEAVE\_ROOM = 4;\\
					MUTE\_USER = 5;\\
					UNMUTE\_USER = 6;\\
					KICK\_USER = 7;\\
				\}\\
				\\
				Typ wyliczeniowy pomocny przy protokole zarządzania pokojem.
				
				
				
				
				
		\end{itemize}	
		
		\item Warstwa transportowa:
		\begin{itemize}
			\item TCP - protokół kontroli transmisji - do komunikacji z serwerem,
			\item UDP - transportowy protokół bezpołączeniowy - do przesyłania próbek z dźwiękiem.
		\end{itemize}
		\item Warstwa sieci:
		\begin{itemize}
			\item IP - podstawowy protokół stosowany w Internecie.
		\end{itemize}
	\end{itemize}
	\section{Schemat bazy danych} 
	\paragraph{} W niniejszym paragrafie zostanie przedstawiony schemat bazy przechowującej dane pokojów.
	\begin{figure}[h]
		\centering
		\includegraphics[scale=0.7]{BD}
		\caption[]{Model tabeli Pokój}
		\label{fig:BD}
	\end{figure}

	Zgodnie z rysunkiem \ref{fig:BD}, baza danych składa się z jednej tabeli opisującej pokój. Id\_pokoju wskazuje na identyfikator konkretnego utworzonego przez użytkownika pokoju. Nazwę pokoju nadaje użytkownik podczas tworzenia pokoju. Hasło dostępu do pokoju oraz hasło administracyjne są wynikiem funkcji skrótu. 
	\section{Diagramy UML}
	\subsection{Diagramy przypadków użycia}
	\paragraph{} W niniejszym paragrafie zostanią przedstawione diagramy UML aktorów oraz ich zadań funkcjonalnych.
	\begin{figure}[H]
		\centering
		\includegraphics[scale=0.9]{UserNP}
		\caption[]{Diagram użytkownika niepołączonego}
		\label{fig:DUN}
	\end{figure}
	\begin{figure}[H]
		\centering
		\includegraphics[scale=0.9]{UserPiAdm}
		\caption[]{Diagram użytkownika połączonego oraz administratora pokoju}
		\label{fig:DUP}
	\end{figure}
	\begin{figure}[H]
		\centering
		\includegraphics[scale=0.9]{UserPK}
		\caption[]{Diagram użytkownika w pokoju konferencyjnym}
		\label{fig:DUW}
	\end{figure}
	\subsection{Diagramy sekwencji}
	\paragraph{} W niniejszym paragrafie zostanią przedstawione diagramy UML przedstawiające proces łączenia się użytkownika z serwerem, tworzenia pokoju konferencyjnego oraz usuwania pokoju konferencyjnego.
	\begin{figure}[H]
		\centering
		\hspace*{-2cm} 
		\includegraphics[scale=0.9]{seq1}
		\caption[]{Diagram połączenia z serwerem}
		\label{fig:seq1}
	\end{figure}
	\begin{figure}[H]
		\centering
		\hspace*{-2cm} 
		\includegraphics[scale=0.9]{seq2}
		\caption[]{Diagram tworzenia pokoju konferencyjnego}
		\label{fig:seq2}
	\end{figure}
	\begin{figure}[H]
		\centering
		\hspace*{-2cm} 
		\includegraphics[scale=0.9]{seq3}
		\caption[]{Diagram usuwania pokoju konferencyjnego}
		\label{fig:seq3}
	\end{figure}
	\subsection{Diagramy stanów}
	\paragraph{} W niniejszym paragrafie zostanią przedstawione diagramy UML przedstawiające przejścia stanów w aplikacji klienckiej oraz serwerowej.
	\begin{figure}[H]
		\centering
		\hspace*{-0cm} 
		\includegraphics[scale=0.9]{stan1}
		\caption[]{Diagram stanów w aplikacji klienckiej}
		\label{fig:kli}
	\end{figure}
	\begin{figure}[H]
		\centering
		\hspace*{-0cm} 
		\includegraphics[scale=0.9]{stan2}
		\caption[]{Diagram stanów w aplikacji serwerowej}
		\label{fig:ser}
	\end{figure}
	\section{Projekt interfejsu graficznego}
	\paragraph*{} W paragrafie zostanie przedstawiony projekt interfejsu graficznego aplikacji.
	\begin{figure}[H]
		\centering
		\hspace*{0cm} 
		\includegraphics[scale=0.9]{1_1.png}
		\caption[]{Interfejs ekranu połączenia}
		\label{fig:gui1}
	\end{figure}
	\begin{figure}[H]
		\centering
		\includegraphics[scale=0.9]{1_2.png}
		\caption[]{Interfejs ekranu głównego aplikacji}
		\label{fig:gui2}
	\end{figure}
	\begin{figure}[H]
		\centering
		\includegraphics[scale=0.9]{1_3.png}
		\caption[]{Interfejs ekranu tworzenia nowego pokoju}
		\label{fig:gui3}
	\end{figure}
	\begin{figure}[H]
		\centering
		\includegraphics[scale=0.9]{1_4.png}
		\caption[]{Interfejs ekranu pokoju podłączony jako administrator}
		\label{fig:gui4}
	\end{figure}
	\begin{figure}[H]
		\centering
		\includegraphics[scale=0.9]{1_5.png}
		\caption[]{Interfejs ekranu pokoju podłączony jako użytkownik}
		\label{fig:gui5}
	\end{figure}
	\begin{figure}[H]
		\centering
		\includegraphics[scale=0.9]{1_6.png}
		\caption[]{Interfejs ekranu ustawień audio}
		\label{fig:gui6}
	\end{figure}

	\section{Najważniejsze metody i fragmenty kodu aplikacji}
	\paragraph*{} W tym paragrafie zostaną przedstawione najważniejsze fragmenty kodu aplikacji.
	\begin{figure}[H]
		\centering
		\hspace*{-1cm} 
		\includegraphics[scale=0.5]{server.png}
		\caption[]{Inicjalizacja serwera TCP}
		\label{fig:server}
	\end{figure}
	\begin{figure}[H]
		\centering
		\includegraphics[scale=0.7]{serverHandler.png}
		\caption[]{Obsługa wiadomości przychodzących i wychodzących w serwerze}
		\label{fig:serverHandler}
	\end{figure}
	\begin{figure}[H]
		\centering
		\includegraphics[scale=0.6]{clientHandler.png}
		\caption[]{Obsługa wiadomości przychodzących i wychodzących u klienta}
		\label{fig:clientHandler}
	\end{figure}
	\begin{figure}[H]
		\centering
		\includegraphics[scale=0.5]{tcpMessage.png}
		\caption[]{Inicjalizacja klienta, wysłanie wiadomości i wyświetlenie odpowiedzi}
		\label{fig:tcpMessage}
	\end{figure}
\end{document}