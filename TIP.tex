\documentclass[12pt,a4paper,notitlepage]{report}
\usepackage{polski}
\usepackage[utf8]{inputenc}
\usepackage[T1]{fontenc}



\usepackage[top=2cm, bottom=2cm, left=3cm, right=3cm]{geometry}

\makeatletter

\newcommand{\linia}{\rule{\linewidth}{0.5mm}}

\renewcommand{\maketitle}{\begin{titlepage}
		
		\vspace*{1cm}
		
		\begin{center}\small
			
			Politechnika Poznańska\\
			
			Wydział Elektryczny\\
			
			Kierunek Informatyka
			
		\end{center}
		
		\vspace{3cm}
		
		\linia
		
		\begin{center}
			
			\LARGE \textsc{\@title}
			
		\end{center}
		
		\linia
		
		\vspace{0.5cm}
		
		\begin{flushright}
			
			\begin{minipage}{5cm}
				
				\textit{ \Large{Autorzy:}}\\
				
				\normalsize{\@author} \par
				
			\end{minipage}
			
		\end{flushright}
		
		\vspace*{\stretch{6}}
		
		\begin{center}
			
			\@date
			
		\end{center}
		
	\end{titlepage}%
	
}

\makeatother

\author{\Large{Szymon Zieliński \\ Oskar Rutkowski} }

\title{\textbf{KakaduVoIP} \\[1cm]  Projekt zespołowy  \\ Telefonia IP}

\begin{document}
	
	\pagenumbering{gobble}
	\maketitle
	\newpage
	\tableofcontents
	\newpage
	\pagenumbering{arabic}
	
	\section{Charakterystyka ogólna projektu}
	\paragraph*{} Aplikacja pozwala na komunikowanie się użytkowników za pomocą mikrofonu oraz głośników w czasie rzeczywistym. Aplikacja jest stworzona w języku Java oraz działa w oparciu o architekturę klient-serwer. Użytkownik po uruchomieniu, jest proszony o podanie nazwy użytkownika (Nick) oraz adresu IP serwera do którego chce się połączyć. Będąc połączonym, użytkownik ma możliwość stworzenia własnego pokoju konferencyjnego lub dołączenia do istniejącego już pokoju. Aby połączyć się z pokojem wybranym z listy, użytkownik musi podać hasło pokoju. Po podaniu prawidłowego hasła, użytkownik ma możliwość rozmowy przy pomocy naciśniętego przycisku. Ma możliwość zmiany komunikowania się w opcjach przewidzianych dla użytkownika będącego na serwerze, na opcję mówienia ciągłego. Użytkownik ma również możliwość wyciszenia innego użytkownika, znajdującego się w pokoju.
	\section{Architektura systemu}
	\paragraph*{} System KakaduVoIP jest oparty o model architektury klient-serwer. Model ten umożliwia podział zadań, serwer zapewnia usługi dla klientów, a klienci zgłaszają żądania obsługi. Serwer korzysta z relacyjnej bazy danych PostgreSQL umiejscowionej w chmurze.
	\section{Wymagania}
	\paragraph*{} W tym paragrafie zostaną omówione wymagania aplikacji funkcjonalne oraz niefunkcjonalne.
	\subsection{Wymagania funkcjonalne}
	\begin{itemize}
		\item użytkownik loguje się do systemu jedynie za pomocą Nazwy użytkownika oraz IP serwera,
		\item użytkownik może zmieniać ustawienia aplikacji względem możliwości komunikowania się, tj. mówienie przez naciśnięcie przycisku lub mówienie ciągłe,
		\item użytkownik może utworzyć pokój konferencyjny, nadając mu hasło,
		\item użytkownik może usunąć pokój konferencyjny,
		\item użytkownik może dołączyć do pokoju wybranego z listy aktualnie dostępnych na serwerze,
		\item użytkownik w pokoju konferencyjnym może wyjść z pokoju,
		\item użytkownik w pokoju konferencyjnym może wyciszyć innego użytkownika, aby go nie słyszeć,
		\item użytkownik w pokoju konferencyjnym może komunikować się z innymi użytkownikami za pomocą przycisku lub mówienia ciągłego,
		\item użytkownik w pokoju konferencyjnym może wyciszyć swój mikrofon lub wyłączyć używanie głośników.
	\end{itemize}
	\subsection{Wymagania niefunkcjonalne}
	\begin{itemize}
		\item aplikacja klienta oraz serwer napisane w języku Java,
		\item aplikacja wieloplatformowa,
		\item szyfrowanie głosu,
		\item do połączenia z serwerem nie jest potrzebna rejestracja użytkownika, jedynym wymogiem jest podanie Nazwy Użytkownika oraz Adresu IP serwera,
		\item do składowania haseł posłuży relacyjna baza danych PostgreSQL,
		\item hasła zabezpieczające pokój będą poddane funkcji haszującej.
	\end{itemize}
	\section{Narzędzia, środowiska, biblioteki}
	\paragraph*{} W niniejszym paragrafie zostaną opisane wykorzystane w aplikacji narzędzia, środowiska oraz biblioteki.
	\paragraph*{} Głównym językiem wykorzystanym do napisania aplikacji jest język Java w wersji 8 Update 162. Wykorzystana jest technika Maven Project, dzięki której możliwa jest praca w różnych środowiskach programistycznych takich jak Intellij oraz Eclipse Oxygen.
	\paragraph*{} Wykorzystanymi frameworkami oraz bibliotekami są:
	\begin{itemize}
		\item Apache MINA - 
		\item Java Media Framework - 
		\item Protocol Buffers - 
		\item java.security.MessageDigest - biblioteka wykorzystana do nakładania funkcji skrótu na hasła,
		\item java.sound.Sampled - biblioteka wykorzystana do przetwarzania dźwięku.
	\end{itemize}
	\paragraph*{} Hasła są przetrzymywane w relacyjnej bazie danych PostgreSQL, działającej w chmurze.
\end{document}